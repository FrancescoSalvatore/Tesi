\chapter{Analisi e testing del sistema}
Le reti che il sistema intende monitorare sono potenzialmente estese e diversificate, sia come natura di dispositivi che come protocolli infrastrutturali. È infatti ragionevole pensare che una rete di home automation impieghi protocolli come Ethernet e 802.15.4 su IPv4, o che un network di sensori per smart cities impieghi tecnologie Sub-1GHz su IPv6. È importante quindi che il sistema di monitoraggio sia flessibile e sia in grado di poter operare con tutte, o buona parte, di queste tecnologie. Potenzialmente l'utente potrebbe registrare decine, centinaia e addirittura migliaia di azioni ed è quindi altrettanto importante che il sistema risponda correttamente alla richiesta di aumento di azioni.
\\In questo capitolo svilupperemo un'analisi del sistema di monitoraggio in merito ai requisiti esposti all'inizio di questo scritto e cercheremo di capire se effettivamente siano stati o meno rispettati.
\\Si cercherà inoltre di dare una fotografia sulle possibili applicazioni d'uso e sugli scenari possibili, oltre che di dare uno sguardo al futuro e alle possibili implementazioni che possono rendere il sistema migliore e quali siano le aree di intervento coinvolte.

\section{Scalabilità, portabilità e interoperabilità}

\section{Scenari d'uso}

\section{Possibili sviluppi futuri}

