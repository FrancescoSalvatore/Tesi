\pagestyle{fancy}
\renewcommand{\chaptermark}[1]{\markboth{#1}{}}
\renewcommand{\sectionmark}[1]{\markright{#1}{}}
\fancyhf{}
\rhead{\rightmark}
\cfoot{\thepage}

\phantomsection
\addcontentsline{toc}{chapter}{Conclusioni}
\chapter*{Conclusioni\markboth{}{Conclusioni}}
Il lavoro svolto ha dimostrato la possibilità di poter raccogliere dati proveniente da fonti eterogenee su reti di natura eterogenea. I risultati ottenuti sono soddisfacenti in questo senso, avendo costruito un efficiente sistema di monitoraggio in grado di osservare smart objects, potenzialmente in aree geografiche ben distinte e anche lontanissime, e pubblicare al di sopra di canali user-friendly i valori ottenuti. Sono stati rispettati tutti i requisiti imposti dall'ambiente dell'Internet of Things, ovvero scalabilità, interoperabilità e portabilità, rendendo possibile astrarre ulteriormente le reti di smart objects dall'utente finale, che è assolutamente agnostico circa le tecnologie e i protocolli in uso a tali reti.
\\La soluzione proposta non è necessariamente la migliore possibile o l'unica praticabile, ma risponde discretamente bene ai requisiti richiesti, con semplicità e chiarezza, utilizzando standard il più possibile aperti e supportati, sia dagli sviluppatori che dalla comunità.
\\La grande flessibilità del sistema permette di poter aggiungere funzionalità e aggiornamenti, garantendo la continua competitività anche a seguito di nuove tecnologie e sviluppi degli smart objects che sono all'ordine del giorno e che saranno sempre più protagonisti del mondo digitale negli anni a venire.
\\Il lavoro compiuto ha quindi gettato le basi per lo sviluppo di strumenti di alto livello che siano in grado di recepire i dati precedentementi osservati e raccolti dal sistema di monitoraggio per elaborarli ulteriormente, oltre che aver dimostrato le vaste possibilità di sviluppo che oggi offre l'Internet of Things. La possibilità di utilizzare linguaggi e paradigmi saldamente radicati nel mondo dei calcolatori che tutti conosciamo dimostra l'enorme flessibilità delle reti di smart objects e del concetto stesso di IoT, che nasce proprio per adattare oggetti di qualunque genere e natura al mondo tecnologico che già esiste ed è alla base della nostra vita quotidiana.