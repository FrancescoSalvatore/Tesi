\pagestyle{fancy}
\renewcommand{\chaptermark}[1]{\markboth{#1}{}}
\renewcommand{\sectionmark}[1]{\markright{#1}{}}
\fancyhf{}
\rhead{\rightmark}
\cfoot{\thepage}

\phantomsection
\addcontentsline{toc}{chapter}{Introduzione}
\chapter*{Introduzione\markboth{}{Introduzione}}

Siamo sempre più circondati dalla tecnologia, che ha oramai pervaso quasi ogni aspetto della nostra vita. Sempre più utilizziamo strumenti elettronici che hanno il compito di rendere la qualità della nostra vita migliore e sempre più lo sviluppo tecnologico ha portato alla riduzione in termini di dimensioni di oggetti un tempo ingombranti e poco maneggevoli. Oggi è normale, per esempio, salire in macchina e connettere il proprio smartphone al sistema di entertainment del veicolo per ascoltare musica o utilizzare il navigatore, oppure archiviare migliaia di documenti in oggetti poco più grandi di una moneta, o ancora indossare orologi in grado di registrare e monitorare i nostri parametri vitali.

Parallelamente alla miniaturizzazione elettronica è maturata un'altra tecnologia che ha invaso il nostro vivere comune: Internet. Il mezzo di comunicazione oggi padrone dello scambio di informazioni è cresciuto esponenzialmente dalla sua nascita negli anni '60, attraversando diverse fasi fino a giungere all'attuale era dell'Internet of Things, l'Internet degli oggetti.
\\\\L'unione delle due cose ha dato vita all'Internet of Things, oggetti interconnessi in grado di raccogliere e scambiare dati tra di loro.
\\\\L'obiettivo è quindi quello di sviluppare un sistema in grado di monitorare questi oggetti e notificare, mediante connettori definiti, i dati raccolti o eventuali cambiamenti di questi. Il sistema deve essere scalabile, data la potenziale mole di oggetti monitorabili, portabile e soprattutto interoperabile, ovvero deve essere in grado di raccogliere informazioni provenienti da fonti architetturalmente differenti.
